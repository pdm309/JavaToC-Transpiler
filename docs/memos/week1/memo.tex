%%%%%%%%%%%%%%%%%%%%%%%%%%%%%%%%%%%%%%%%%
% Memo
% LaTeX Template
% Version 1.0 (30/12/13)
%
% This template has been downloaded from:
% http://www.LaTeXTemplates.com
%
% Original author:
% Rob Oakes (http://www.oak-tree.us) with modifications by:
% Vel (vel@latextemplates.com)
%
% License:
% CC BY-NC-SA 3.0 (http://creativecommons.org/licenses/by-nc-sa/3.0/)
%
%%%%%%%%%%%%%%%%%%%%%%%%%%%%%%%%%%%%%%%%%

\documentclass[letterpaper,11pt]{../texMemo} % Set the paper size (letterpaper, a4paper, etc) and font size (10pt, 11pt or 12pt)

\usepackage{parskip} % Adds spacing between paragraphs
\usepackage{hyperref}

\setlength{\parindent}{15pt} % Indent paragraphs

%----------------------------------------------------------------------------------------
%	MEMO INFORMATION
%----------------------------------------------------------------------------------------

\memoto{Randy Shepherd} % Recipient(s)

\memofrom{Jason Yao} % Sender(s)

\memosubject{Team 3: Panic n' OOP Weekly Memo} % Memo subject

\memodate{\today} % Date, set to \today for automatically printing todays date

\logo{\includegraphics[width=0.3\textwidth]{../img/panic.png}} % Institution logo at the top right of the memo, comment out this line for no logo

%----------------------------------------------------------------------------------------

\begin{document}

\maketitle % Print the memo header information

%----------------------------------------------------------------------------------------
%	MEMO CONTENT
%----------------------------------------------------------------------------------------
\section{In-class Work Update}
Two in-class assignments have currently been given and completed by our team. Our approach was to split the group into two sub-teams for each task, and rotate members through so that they get paired with different people every time.

\subsection{xtc-in-class}
\href{https://github.com/JasonYao/xtc-in-class/}{Github repo}, 
\href{https://github.com/nyu-oop-fall16/xtc-in-class/pull/9}{Pull Request}

Team update: All members contributed to the project in two sub-teams:
\\{\bf Team 1:} Amritanshu Kajaria, Jason Yao
\\{\bf Team 2:} Jenna Denker, Paul Merritt, James Zhang
\\Team did well, though a few issues with git did occur. As the semester progresses these issues should pop up with less frequency as people acclimate to using git.

\subsection{java-lang-in-class}
\href{https://github.com/JasonYao/java-lang-in-class}{Github repo}, 
\href{https://github.com/nyu-oop-fall16/java-lang-in-class/pull/6}{Pull Request}

Team update: All members contributed to the project in two sub-teams:
\\{\bf Team 1:} Jenna Denker, Amritanshu Kajaria
\\{\bf Team 2:} Paul Merritt, Jason Yao, James Zhang

Both teams worked in parallel to manually build out the C++ program from a given Java source well. Continuing issues with git, but now all members understand how to use branching, merging, and pulling.

\section{Semester Project Update}
Currently the team has met once outside of class specifically for the semester project, and we have introduced team members to the general git flow that will be in use for the semester.

\subsection{Team Learning}
Team members will work in their own branches, before having them merge to a "dev" branch, where any merge conflicts become solved there, before the clean dev branch will then be merged back to master.

\subsection{Work Completed}
All team members have pulled the semester transpiler repo, and are reading through the example xtc implementations shown to familiarise themselves with the task.

\subsection{Work Scheduled}
The team is set to have its next meeting on Monday, 10th October, 2016 to set up all milestones and sub-tasks in a clear manner, allowing team members in the future to work on modular sub-issues in parallel. Github's project management tooling along with its issue tracker will be used for this purpose.

%----------------------------------------------------------------------------------------
\end{document}